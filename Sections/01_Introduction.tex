%=== CHAPTER ONE (1) ===
%=== INTRODUCTION ===

\chapter{Introduction}
\begin{spacing}{2.0}

Chess is a game of strategy and skill that has captivated players and spectators for centuries. As the popularity of chess continues to grow, there is a growing interest in understanding the characteristics and abilities of individual chess players. The ability to identify chess players based on their playing style and patterns can provide valuable insights into player profiling, performance analysis, and talent identification. However, manual identification of chess players is a laborious and subjective task, often limited by human biases and errors. Recent advancements in machine learning, particularly neural networks, offer promising opportunities to automate the identification process and extract meaningful patterns from large volumes of chess data.

This paper aims to explore the identification of chess players through the application of neural networks. The research is motivated by the desire to develop an automated system capable of accurately recognizing individual chess players based on their unique playing styles and strategies. By leveraging neural networks, this study intends to enhance our understanding of player characteristics and facilitate practical applications within the chess community.

Positioning this research within the field of machine learning, chess analysis, and player profiling, the investigation will follow the research onion framework. This framework will guide the systematic exploration of various layers of analysis, including data collection, preprocessing, neural network architecture, training processes, and evaluation metrics.

The background to this research theme lies in the limitations of manual identification methods, which are time-consuming, subjective, and prone to human biases and errors. Previous attempts to automate player identification have relied on rule-based systems or statistical techniques, which often fail to capture the complex patterns and nuances of individual playing styles. The advancements in machine learning and neural networks provide a compelling opportunity to overcome these limitations and offer a more accurate and objective means of identifying chess players.

The hypothesis for this research posits that by training neural networks on extensive datasets of chess games, it is possible to develop an automated system that accurately identifies individual chess players based on their unique playing styles and strategies.

The research aim is to explore the application of neural networks for the identification of chess players, with the purpose of developing an automated system capable of effectively recognizing and differentiating chess players based on their playing patterns and strategies. The intended outcomes include enhancing our understanding of player characteristics, contributing to player profiling and analysis, and enabling practical applications such as player rankings, talent identification, and personalized coaching within the chess community.


\end{spacing}
%=== END OF CHAPTER ONE ===
\newpage


